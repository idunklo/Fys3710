\documentclass[a4paper,norsk,12pt,oneside]{article}
% To use norwegian
\usepackage[utf8]{inputenc}
\usepackage[T1]{fontenc}
\usepackage[norsk]{babel}
% math
\usepackage{amsmath}
\usepackage{amsfonts}
\usepackage{amssymb}

\usepackage{graphicx} % images
\usepackage{float}
\usepackage{enumerate}
\usepackage{fancyvrb} % code
\usepackage{algorithm2e} % algorithm
% For listing
\usepackage{listings} % code with color
\usepackage{courier}
\usepackage{caption}
\usepackage{color}

\title{Fys3710, Disposisjon for semesteroppgave}
\author{Idun Kløvstad}
\date{\today}  

% Command for L'Hopital's rule (requires extarrows-package)
\newcommand{\Heq}[1]{\xlongequal[\mathrm{L'H}]{\left[#1\right]}}

% Double underline
\newcommand{\dul}[1]{\underline{\underline{#1}}}

% Custom operators
\DeclareMathOperator{\nul}{Nul\,}
\DeclareMathOperator{\Proj}{Proj\,}
\DeclareMathOperator{\Sp}{Sp\,}
\DeclareMathOperator{\res}{Res\,}
\DeclareMathOperator{\Log}{Log\,}

% Allow displayed page breaks
\allowdisplaybreaks

% Settings for listings
\definecolor{dkgreen}{rgb}{0,0.6,0}
\definecolor{gray}{rgb}{0.5,0.5,0.5}
\definecolor{mauve}{rgb}{0.58,0,0.82}

\lstset{%
  %language=python,                     % the language of the code
  basicstyle=\footnotesize\ttfamily,   % the size of the fonts that are used for the code
  numbers=left,                        % where to put the line-numbers
  numberstyle=\tiny,                   % the style that is used for the line-numbers
  stepnumber=2,                        % the step between two line-numbers. If it's 1, each line will be numbered
  numbersep=5pt,                       % how far the line-numbers are from the code
  extendedchars=true,
  %backgroundcolor=\color{white},       % choose the background color. You must add \usepackage{color}
  %showspaces=false,                    % show spaces adding particular underscores
  showstringspaces=false,              % underline spaces within strings
  %showtabs=false,                      % show tabs within strings adding particular underscores
  %frame=single,                        % adds a frame around the code
  frame=b,                             % adds a line at the bottom
  %rulecolor=\color{black},             % if not set, the frame-color may be changed on line-breaks within not-black text (e.g. commens (green here))
  tabsize=2,                           % sets default tabsize to 2 spaces
  %captionpos=b,                        % sets the caption-position to bottom
  breaklines=true,                     % sets automatic line breaking
  breakatwhitespace=false,             % sets if automatic breaks should only happen at whitespace
  %title=\lstname,                      % show the filename of files included with \lstinputlisting; also try caption instead of title
  keywordstyle=\color{blue},           % keyword style
  commentstyle=\color{dkgreen}\textit, % comment style
  stringstyle=\color{mauve},           % string literal style
  %escapeinside={\%*}{*)},              % if you want to add LaTeX within your code
  %morekeywords={*,...},                % if you want to add more keywords to the set
  xleftmargin=-20pt,
  xrightmargin=-20pt,
  framexleftmargin=19pt,
  framexbottommargin=4pt,
  framexrightmargin=21pt,
} 

% caption for listing
\newlength{\mycapwidth}\setlength{\mycapwidth}{\textwidth}
\addtolength{\mycapwidth}{75pt}
\DeclareCaptionFont{white}{\color{white}}
\DeclareCaptionFormat{listing}{\colorbox[cmyk]{0, 0, 0, 0.8}
    {\parbox{\mycapwidth}{\hspace{15pt}#1#2#3}}}
\captionsetup[lstlisting]{format=listing,labelfont=white,textfont=white, 
    singlelinecheck=false, margin=-40pt, font={bf,footnotesize}}


\begin{document}
\maketitle
\newpage 

\begin{abstract}
    En kort oppsummering av oppgaven. 

    Bruk av numeriske metoder innenfor nevrovitenskap. Hvorfor Hodkin Huxley metoden er viktig
    og hva den går ut på. 

\end{abstract}

\section*{Introduksjon} 

\begin{itemize}

    \item En liten generell bit om konstruksjoner av metoder for å regne på deler av nervesystemet. 

    \item Litt historie rundt Hodkin Huxley modellen. Første suksessfulle eksempelet på å kombinere eksperimentelle 
        studier med programering innen neurovitenskap. 

    \item Kort forklaring av Hodkin Huxley modellen. Den simulerer en forplanting av en nerveimpuls 
        langs et axon.

\end{itemize}

\section*{Teori}

\begin{itemize}

    \item Elekstrisk aktivitet i neuroner og litt om nerveimpulser. 

    \item Hvordan modellen utvikles.

        \begin{itemize}

            \item Tilsvarer en elektrisk krets med ionestrøm. 

                \begin{itemize}

                    \item natrium  

                    \item kalium 

                    \item leak (lekkasje på norsk?)
                \end{itemize}

            \item Hvordan disse settes sammen til en komplett modell

        \end{itemize}

\end{itemize}

\section*{Diskusjon}

\begin{itemize}

    \item Hvordan bruke modellen

        \begin{itemize}

            \item Hvilken informasjon kreves

            \item Hvilken informasjon kan vi hente ut

        \end{itemize}

    \item Nøyaktighet og presisjon

    \item Fordeler og ulemper ved modellen

    \item Videreutviklinger og variasjoner av modellen

\end{itemize}

\section*{Avslutning av konklusjon}

\begin{itemize}

    \item Kort oppsumering av modellen med fordeler og ulemper

    \item Hva som gjør modellen viktig, basert på informasjon fra teori og diskusjon. 

    \item Hvor går veien videre?

\end{itemize}

\end{document}                                             
                                                  
